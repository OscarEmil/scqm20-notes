\documentclass[a4paper]{article}
\author{Oscar Emil Sommer}
\input{../preamble}
\input{../macros}

\fancyhead[L]{\textbf{Frustrated Magnetism}}


\begin{document}

Colloqium on Strongly interacting QCD Matter at finite temperature and Density
by Tetsuo Hatsuda


Quarks also have colour charge; fundamental representation of SU(3)


Gluons are adjoint representation of SU(3), so 8 types of gluons



Quantum Chromo dynamics (QCD)
$\SU(3)$ gauge theory for colour charges

\[ \mathcal{L}=-\frac{1}{4} G_{\mu \nu}^a G_a^{\mu \nu} +\bar{q}\gamma^\mu
(i\partial_\mu -gt^aA^a_\mu)q - m\bar{q}q\]
where 
\[ G_{\mu\nu}^a=\partial_\mu A^a_\nu -\partial_\nu A_\mu ^a +g f_{abc} A^b_\mu
A^c_\nu\]

Important concepts are \emph{color confinement} and \emph{assymptotic freedom}.
Running coupling $\alpha_s=g^2/4\pi$ is a function of energy scale and goes to
0 at large energy -> assymptotically free. At low energy coupling is very
strong, and the charges are bound in the low in the low energy regime; can
never see colour on its own.

How do we calculate stuff with high coupling? Lattice QCD.
Use monte-carlo on fine lattice over many configuration

Potential is numerically shown to be $V(r)=-c/r +Kr$.


\subsection{Spontanous chiral symmetry breaking}
Quark masses $m_{u,d}\sim 3\text{MeV}$. However nucleon masses are closer to
900 MeV, how does this arise? In hadronic phase, there is a chiral symmetry
breaking giving all quarks a mass of $m_eff =300MeV$.

\subsection{Phase diagram of QCD}

Hadronic phase, Quark-gluon plasma, colour superconductor and an in-between 
\end{document}
