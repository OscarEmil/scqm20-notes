\documentclass[a4paper]{article}
\author{Oscar Emil Sommer}
\makeatletter
\usepackage{amsfonts}
\usepackage{amsmath}
\usepackage{amssymb}
\usepackage{amsthm}
\usepackage{bm}
\usepackage[english]{babel}
\usepackage{booktabs}
\usepackage[margin = 10pt, font = small, labelfont = bf]{caption}
\usepackage{fancyhdr}
\usepackage{graphicx}
\usepackage{xcolor}
\usepackage{mathtools}
\usepackage{microtype}
\usepackage{multirow}
\usepackage{pdflscape}
\usepackage{pgfplots}
\usepackage{siunitx}
\usepackage{tabularx}
\usepackage{tikz}
\usepackage{sectsty}
\usepackage{derivative}
\usepackage{braket}
\usepackage[colorlinks=true
   ,urlcolor=blue
   ,anchorcolor=blue
   ,citecolor=blue
   ,filecolor=blue
   ,linkcolor=blue
   ,menucolor=blue
   ,linktocpage=true
   ,pdfa=true
]{hyperref}

\pagestyle{fancyplain}
\fancyhead[R]{Oscar Emil Sommer}
\fancyfoot{}
\fancyfoot[C]{\thepage}

\renewcommand{\@dotsep}{10000} %No dots in ToC
\allsectionsfont{\bfseries\sffamily} %Sections are bold sans serif



% Theorem-like environments
\theoremstyle{definition}
\newtheorem*{axiom}{Axiom}
\newtheorem*{aprx}{Approximation}
\newtheorem*{claim}{Claim}
\newtheorem{theorem}{Theorem}[section]
\newtheorem{corollary}[theorem]{Corollary}
\newtheorem{definition}{Definition}[section]
\newtheorem{conjecture}{Conjecture}
\newtheorem*{example}{Example}
\newtheorem*{exercise}{Exercise}
\newtheorem*{fact}{Fact}
\newtheorem*{remark}{Remark}
\newtheorem*{method}{Method}
\newtheorem{lemma}[theorem]{Lemma}
\newtheorem{proposition}[theorem]{Proposition}


\newtheorem*{principle}{Principle}
\setcounter{tocdepth}{2}
\makeatother

%Upright symbols
\renewcommand{\i}{\mathrm{i}}
\newcommand{\e}{\mathrm{e}}
\newcommand{\p}{\partial}
%Vector formatting
\renewcommand{\Vec}[2]{\left(\begin{array}{c} {#1} \\ {#2} \end{array}\right)}


%Common sets
\newcommand{\R}{\mathbb{R}}
\newcommand{\Z}{\mathbb{Z}}
\newcommand{\N}{\mathbb{N}}
\newcommand{\Q}{\mathbb{Q}}
\newcommand{\C}{\mathbb{C}}
\newcommand{\M}{\mathcal{M}}

%Common Groups
\newcommand{\SU}{\mathrm{SU}}
\newcommand{\SL}{\mathrm{SL}}
\newcommand{\GL}{\mathrm{GL}}
\newcommand{\Orth}{\mathrm{O}}
\newcommand{\SO}{\mathrm{SO}}
\newcommand{\U}{\mathrm{U}}
\renewcommand{\Im}{\operatorname{Im}}
\renewcommand{\Re}{\operatorname{Re}}
\renewcommand{\O}{\mathcal{O}}

\newcommand{\su}{\mathfrak{su}}
\newcommand{\gl}{\mathfrak{gl}}
\newcommand{\so}{\mathfrak{so}}
%Common stylistic variable

\newcommand{\ep}{\epsilon}
\newcommand{\var}[1]{\delta #1\,}
\newcommand{\dd}[1]{\mathrm{d}#1\,}
\newcommand{\oes}[1]{\textcolor{red}{#1}}
\newcommand{\oesimp}[1]{\textcolor{blue}{#1}}
\newcommand{\vb}{\mathbf}
\newcommand{\ketbra}[2]{\ket{#1}\!\!\bra{#2}}
\newcommand{\Tr}{\operatorname{Tr}}
\newcommand{\up}{\uparrow}
\newcommand{\down}{\downarrow}


\fancyhead[L]{\textbf{Frustrated Magnetism}}


\begin{document}

Colloqium on Strongly interacting QCD Matter at finite temperature and Density
by Tetsuo Hatsuda


Quarks also have colour charge; fundamental representation of SU(3)


Gluons are adjoint representation of SU(3), so 8 types of gluons



Quantum Chromo dynamics (QCD)
$\SU(3)$ gauge theory for colour charges

\[ \mathcal{L}=-\frac{1}{4} G_{\mu \nu}^a G_a^{\mu \nu} +\bar{q}\gamma^\mu
(i\partial_\mu -gt^aA^a_\mu)q - m\bar{q}q\]
where 
\[ G_{\mu\nu}^a=\partial_\mu A^a_\nu -\partial_\nu A_\mu ^a +g f_{abc} A^b_\mu
A^c_\nu\]

Important concepts are \emph{color confinement} and \emph{assymptotic freedom}.
Running coupling $\alpha_s=g^2/4\pi$ is a function of energy scale and goes to
0 at large energy -> assymptotically free. At low energy coupling is very
strong, and the charges are bound in the low in the low energy regime; can
never see colour on its own.

How do we calculate stuff with high coupling? Lattice QCD.
Use monte-carlo on fine lattice over many configuration

Potential is numerically shown to be $V(r)=-c/r +Kr$.


\subsection{Spontanous chiral symmetry breaking}
Quark masses $m_{u,d}\sim 3\text{MeV}$. However nucleon masses are closer to
900 MeV, how does this arise? In hadronic phase, there is a chiral symmetry
breaking giving all quarks a mass of $m_eff =300MeV$.

\subsection{Phase diagram of QCD}

Hadronic phase, Quark-gluon plasma, colour superconductor and an in-between 
\end{document}
