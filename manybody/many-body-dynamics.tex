\documentclass[a4paper]{article}

\input{../preamble}
\author{Oscar Emil Sommer}
\usetikzlibrary{calc}
\input{../macros}
\begin{document}
Many body Quantum dynamics by Vedika Khemani
\section{Emergence: More is different}
The idea of condensed matter is \emph{emergence}, coined by Anderson in 1972:
More is \emph{different}; There are phenoma you can only see when you consider
the collective properties of many bodies.

How can we deal with $N\simeq 10^{23}$ particles? There are two key
frameworks, Quantum statistical Mechanics and Emergent Quasiparticles.

In Quantum statmech you have system interchanging particles/energy with a bath
of much greater size. The system reaches Thermal equilbrium at late times,
obtained from max entropy principle. Hence $\rho^S(t)\to
\rho^S_\mathrm{eq}(T,\mu)$ and the quantitites of interest are
$\braket{O}=\Tr\left[\rho_\mathrm{eq} O\right]$. This is built on the
\emph{ergodic hypothesis} of chaotic systems.

Quasi particles are the low energy excitations from the ground state of
a system. In particular, these only interact weakly.

\subsection{Approach to many-body physics.}
Traditionally we study Hamiltonian time evolution with a time independent
Hamiltonian $H(t)=H$. Recently we have modified this paradigm by interrupting
the unitary time evolution by non-unitary measurements. Generally we study
ground or equilibrium states, or small deviations from these at low energies.
This has recently been changed to study dynamics of initial states that are not
low-energy in any sense. Traditionally we study correlation functions, order
parameters for characterizing universal phases and phase transitions, but
recently we've moved on to Quantum entanglement, Quantum complexity etc.

The focus of this seminar is the \emph{dynamics} of 
\begin{itemize}
    \item Strongly interacting
    \item Isolated (Unitary time evolution), so $\rho(t+\delta
        t)=U\rho(t) U^\dagger$.
    \item Highly excited (No approaches based on quasiparticles)
    \item Many body systems (spins, cold atoms, qubits, etc.)
\end{itemize}


An outline of the lecture is:
\begin{itemize}
    \item Eigenstate thermalisation hypothesis
    \item Many body localisation: Breakdown of thermalisation, Useful picture
        called $\ell$-bit picture
    \item Localisation protected quantum order
    \item Quantum Chaose and Out of time ordered commutators.
\end{itemize}
\begin{itemize}
    \item Q.statmech MBL, Nadkishore Huse, Ann. Rev CMP (2015)
    \item MBL: Abanin,Altman,Bloch, Serbyn. Rev. Mod. Phys (2019)
    \item Thermalisation, ETH: D'Alessio,Kafri, Polkovnikov, Rigol, Adv. Phys (2016)
    \item Localisation protected quantum order, time crystals: Khemani, Moessner,Sondhi,
arXiv 1910.10745 (2019)
\end{itemize}
\begin{definition}[classes of Models]
    Will be working with models of the following types 
    \begin{description}
        \item[Time ind Hamiltonians] Conserved energy, has eigenstates $U=\e^{-i
            Ht}$.
        \item[Floquet] Periodically driven $H(t+T)=H(t)$,
            $U(nT)=\left[U(T)\right]^n$. Has eigenstates $\ket{\alpha}$ with
            $U(T)\ket{\alpha}=\e^{i\alpha}\ket{\alpha}$.
        \item[Random unitary circuits] Layers of random 1 and 2 body gates. Has
            unitary dynamics with locality.
    \end{description}
\end{definition}

Why are we changing approach now? Because experimentalists are designing
artificial many-body quantum system, including ultracold atoms, superconducting
qubits, trapped ions, cavity QED. 

\section{Eigenstate thermalisation hypothesis}
\begin{definition}
    Can an isolated strongly interacting many body system act as its own `bath'
    and bring its subsystems to thermal equilibirum?
\end{definition}
To answer the question we need to answer fundamental questions like
What does thermal equilibrium mean in this context?
How is thermal equilbrium reached when it is reached, and can it be avoided?

The first example of a system that can be many-body localised and fail to go to
equilibrium was Andersons 1958 paper. 
It turns out there are two distinct possibilities: Either a system can
thermalise or it will be many-body localised (Can also be Integrability, scars
but these are believed to be \emph{fine tuned}). There is
a kind of quantum phase transition between MBL and thermalised.

\subsection{Thermalisation in isolation}
Actually can have intermidate systems between MBL and thermalising. 
Under unitary time evolution system \emph{remembers} all details, as
$\ket{\psi(t)}=U(t)\ket{\psi_0}$ is reversible. In a thermalising system the
memory gets \emph{scrambled} into completely non-local degrees of freedom. So
even if it is formally reversible, in practice we can only probe small local
subsystems. These subsystems $A$ of $A\cup B$ can therefore lose memory of its
initial conditions. A thermal state is a state $\rho_A=\Tr_B \rho_{AB}$ which in
the limit of large system $|A\cup B|\to \infty$ looks like an equilibrium state
with the bath traced out $\rho_A\to \Tr_B
\rho_\mathrm{eq} (T,\mu)$.
\subsection{Equilibrium states and conservation laws}
What is the right equilibrium state for the subsystems to thermalise to? There
is no reservoir, so which conserved quantities are interchanged? For systems
with only energy conservation it would be Grand Canonical Ensemble
$\rho_G=\frac{1}{Z} \e^{-\beta H}$, with energy and particle number it would be
the grand canonical ensemble $\rho_\mathrm{GC}=\frac{1}{Z}\e^{-\beta(H-\mu N)}$,
and in general could have an arbitrary number of conserved quantities like
integrable systems, where $\rho_\mathrm{GGE}=\frac{1}{Z}\e^{-\beta(H-\mu_1
N_1-\mu_2 N_2\dots)}$. Note that not all conserved operators should be accounted
for. 
\begin{example}
    For any time independent hamiltonian $H$ the eigenstates $\ket{\alpha}$
    produce an exponential number of conserved operators through there
    projectors $P_\alpha=\ketbra{\alpha}{\alpha}$, as these commute with the
    hamiltonian $[H,P_\alpha]=0$. These projectors are not local in any sense.
\end{example}
The only operators we include are the superposition of local operators $N=\sum_i
c_i O_i$. The notion of locality is not rigoursly defined, and depend on what
kind of observables are. Examples could be finite range, exponentially decaying,
power law decaying etc.

What fixed the lagrange multipliers? There is no reservoir to fix $\beta$ etc.
In fact it will be overall density of your conserved quantity. So if start with
$\rho_0$, then $\rho_0(t)=U\rho_0^{AB} U^\dagger$ to reach thermal equilibrium
we need that 
\[ \lim_{t\to infty}\lim_{|AB|\to\infty} \Tr_B \rho_0^{AB}(t)=\rho_{eq}^{(A)},
\]
To set $\beta$ need $\braket{E}=\Tr(H\rho_0)$, and $\Delta E
=\sqrt{\Tr(H^2\rho_0)-\left(\Tr H\rho_0\right)^2}\propto V^\alpha$ with
$\alpha<1$, so that $\Delta E/\braket{E}\to 0$ as volume to $\infty$. Generally
most initial states will have $\alpha=1/2$, but may be counterexamples.

\begin{example}
    Let $H\ket{\alpha}=\alpha\ket{\alpha}$ be the eigenstates of a hamiltonian.
    Now the time depedence of any expected value, given the initial state
    $\ket{\psi_0}=\sum_\alpha c_\alpha \ket{\alpha}$, will be 
    \[
    \braket{\psi(t)|O|\psi(t)}=\sum_\alpha |c_\alpha|^2 O_\alpha
    +\sum_{\alpha\neq\beta} c_\alpha c^*_\beta \braket{\beta|O|\alpha}
    \e^{-i t(E_\alpha-E_\beta)}
    \]
    The diagonal elements is the \emph{diagonal ensemble} and is responsible of
    the equilbrium value of the observable. There are two distinct processes at
    play
    \begin{description}
        \item[Equilibration] The expectation value of operators tend toward
            equilbrium value, as the off-diagonal ensemble loose coherence and
            average to 0
        \item[Thermalisation] The equilibrium values are given by the thermal
            ensemble $\Tr O_A\rho_{AB}^\mathrm{eq}$.
    \end{description}
\end{example}
\subsection{Eigenstate thermalisation hypothesis}
If all \emph{well defined} initial states reach thermal equilibrium then
eigenstates of $H$ must be thermal:
\begin{align*}
    H\ket{n}&=E_n\ket{n}\\
    \Tr_B\ketbra{n}{n} &=\Tr_B \frac{\e^{-\beta_n H}}{Z}
\end{align*}
Single-eigenstates are well-defined microcanical ensemble. 
Strong version: All states are thermal.
Weak version: Almost all states are thermal (exceptions are scars).
\begin{example}[How to define temperature]
    Plot entropy or DoS  as function of energy density, will get gaussian for
    local hamiltonian. The slope of this curve is then $\beta$. Note that
    infinite temperature is the energy density is in  middle of the spectrum.
\end{example}
\subsection{Volume law entanglement}
It is very illuminating to think of entanglement as an observable. For subsystem
$A$, with reduced density $\rho_A=\Tr_B \rho$, then the entanglement entropy is
$S_A=-\Tr [\rho_A\log \rho_A]$. Now ETH means that this takes on the the
thermal value $S_A= V_a s_\mathrm{th}+..$ scaling as volume of $A$.
Note do not need to think of $B$ as reservoir of anything, just as something to
get entangled with! The conserved quantities are not essential. For Floquet
systems for example, there is no conservation of energy, so
$\rho^\mathrm{eq}_\mathrm{Floquet}\propto 1$. 
\section{Which systems reach equilibrium?}
Do all generic systems reach thermal equilibrium? Most do, but we are aware of
one general exception \emph{localisation}. Occur in systems without
translational invariance, often they are disordered, but this isn't required. 

\begin{definition}
    systems retain \emph{local} memory of initial conditions to infinitely late
    times! 
\end{definition}
New kind of transition between MBL to thermalising phases. MBL can stabilise new
kinds of non-eq order (eg time crystals)
An experimental verification is in Choi et al. Science (2016), where atoms
retain memory of what region of a trap they occupy.
\subsection{What do we know about MBL transitions?}
\begin{itemize}
    \item Anderson: Non-interacting electrons can be localised by disorder
        (strong in 3D, arbitrarily weak is enough in 1D and 2D)
    \item MBL: Generalisation of anderson localisation to case of interactions
    \item Existence of the MBL phase
        \begin{itemize}
            \item All orders in perturbation theory in small interaction
                strength,in any dimension (2006)
            \item Almost proof including non-perturbative effects in one
                dimensional lattice models with exponentially decaying
                interactions. (It is believed that in higher dimensions it may
                eventually be unstable to non-perturbative effects)
            \item Lots of open questions (possible non-perturbative
                instabilities in higher dimensions, with longer interactions..)
                (de Roeck, Huveneers).
        \end{itemize}
    \item Lots of numerical evidence of the thermal phase, but no proof
\end{itemize}
\begin{example}[Single-particle Anderson Localisation]
    A simple model for the Anderson Localisation is non-interacting fermions
    hopping in random potential
    \[ H=\sum_i h_i c^\dagger_i c_i +J(c_i^\dagger c_{i+1} +\mathrm{h.c.}),\]
    with $h_i\in [-W,W]$. Locator expansion $J\ll W$ means that wavefunctions
    will be localised exponentiall around each site $|\phi(r)|^2\sim
    e^{-r/\xi}$.
\end{example}
\begin{example}[Localisation with interactions]
    Do a Jordan-Wigner transformation to a spin system (Up is occupied and down
    is empty), and add interactions to
    obtain hamiltonian
    \[
        H = \sum_i h_i\sigma_i^z +J\sum_i\left( \sigma_i^x\sigma_{i+1}^x
            +\sigma_i^y\sigma_{i+1}^y +\sigma_i^z\sigma_{i+1}^z\right)
    \]
    There is a lot of numerical exploration of this system, and get a Energy
    density vs disorder strength $W$ phase diagram. At sufficiently high
    disorder appear to see localised phase, but a few problems with numerics.
\end{example}
\subsection{$\ell$-bit Picture of Many body localisation}
MBL is sort of an emergent integrability. An integrabel system has an extensive
number of conserved quantities. For previous model, if turn off $J=0$,
eigenstates are just spin chains, which are not thermal, and have non-local
correlations; they violate
ETH. There
is no transport, and the only dynamics is Larmor precession. This can be thought
of as arising from the fact that we have an extensive number of constants of
motion $\{\sigma_i^z\}$, and $[\sigma^z_i,\sigma_i^z]=0$.
What happens when we add interactions? $H=\sum_i h_i\sigma_i^z +J\sum_i
\boldsymbol{\sigma}_i\cdot\boldsymbol{\sigma}_{i+1}$ for $J<<W$ there is a quasi local
unitary operator transforming $H=\sum_i \tilde{h}_i \tau_i^z +\sum_ij
\tilde{J}_{ij}\tau_i^z\tau_j^z+\dots$, where $\tau_i^z=U^\dagger \sigma_i^z U$
are an extensive number of conserved quantities. The special aspect is that the
unitary transformation is \emph{quasi-local} (exponentially decay) converting from physical bits
($p$-bits) to local bits ($\ell$-bits), so the hamiltonian is still local. The
locality follow from the fact that the ground state is gapped, which implies
there is a local unitary transforming from the initial ground state to the new
ground state. The local unitary is a finite depth circuit $V$. The finite depth
means that we may cut off the tails of our dressed operator at some finite
length.
\begin{example}[Ising model]
    \[
    H=J\sum_i Z_i Z_{i+1} + h\sum_i X_i
\]
In the $J=0$ limit the ground state is eigenstate of $X_i$. For $J\neq 0$ can
deform operators $\tilde{X}_i$ to become eigenoperators of new ground state. At
phase transition this deformation has infinite correlation length, so becomes
useless. For MBL it is same pictur, except it is not just the ground state,
but rather all the states for which this occurs. 
\end{example}
\begin{example}[Area law entanglement for MBL eigenstates]
    In MBL generic eigenstate only area of each subsystem is entangled with rest
    of system, so $S_A \propto \partial A$.
\end{example}

\end{document}
