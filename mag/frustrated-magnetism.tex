\documentclass[a4paper]{article}
\author{Oscar Emil Sommer}
\makeatletter
\usepackage{amsfonts}
\usepackage{amsmath}
\usepackage{amssymb}
\usepackage{amsthm}
\usepackage{bm}
\usepackage[english]{babel}
\usepackage{booktabs}
\usepackage[margin = 10pt, font = small, labelfont = bf]{caption}
\usepackage{fancyhdr}
\usepackage{graphicx}
\usepackage{xcolor}
\usepackage{mathtools}
\usepackage{microtype}
\usepackage{multirow}
\usepackage{pdflscape}
\usepackage{pgfplots}
\usepackage{siunitx}
\usepackage{tabularx}
\usepackage{tikz}
\usepackage{sectsty}
\usepackage{derivative}
\usepackage{braket}
\usepackage[colorlinks=true
   ,urlcolor=blue
   ,anchorcolor=blue
   ,citecolor=blue
   ,filecolor=blue
   ,linkcolor=blue
   ,menucolor=blue
   ,linktocpage=true
   ,pdfa=true
]{hyperref}

\pagestyle{fancyplain}
\fancyhead[R]{Oscar Emil Sommer}
\fancyfoot{}
\fancyfoot[C]{\thepage}

\renewcommand{\@dotsep}{10000} %No dots in ToC
\allsectionsfont{\bfseries\sffamily} %Sections are bold sans serif



% Theorem-like environments
\theoremstyle{definition}
\newtheorem*{axiom}{Axiom}
\newtheorem*{aprx}{Approximation}
\newtheorem*{claim}{Claim}
\newtheorem{theorem}{Theorem}[section]
\newtheorem{corollary}[theorem]{Corollary}
\newtheorem{definition}{Definition}[section]
\newtheorem{conjecture}{Conjecture}
\newtheorem*{example}{Example}
\newtheorem*{exercise}{Exercise}
\newtheorem*{fact}{Fact}
\newtheorem*{remark}{Remark}
\newtheorem*{method}{Method}
\newtheorem{lemma}[theorem]{Lemma}
\newtheorem{proposition}[theorem]{Proposition}


\newtheorem*{principle}{Principle}
\setcounter{tocdepth}{2}
\makeatother

%Upright symbols
\renewcommand{\i}{\mathrm{i}}
\newcommand{\e}{\mathrm{e}}
\newcommand{\p}{\partial}
%Vector formatting
\renewcommand{\Vec}[2]{\left(\begin{array}{c} {#1} \\ {#2} \end{array}\right)}


%Common sets
\newcommand{\R}{\mathbb{R}}
\newcommand{\Z}{\mathbb{Z}}
\newcommand{\N}{\mathbb{N}}
\newcommand{\Q}{\mathbb{Q}}
\newcommand{\C}{\mathbb{C}}
\newcommand{\M}{\mathcal{M}}

%Common Groups
\newcommand{\SU}{\mathrm{SU}}
\newcommand{\SL}{\mathrm{SL}}
\newcommand{\GL}{\mathrm{GL}}
\newcommand{\Orth}{\mathrm{O}}
\newcommand{\SO}{\mathrm{SO}}
\newcommand{\U}{\mathrm{U}}
\renewcommand{\Im}{\operatorname{Im}}
\renewcommand{\Re}{\operatorname{Re}}
\renewcommand{\O}{\mathcal{O}}

\newcommand{\su}{\mathfrak{su}}
\newcommand{\gl}{\mathfrak{gl}}
\newcommand{\so}{\mathfrak{so}}
%Common stylistic variable

\newcommand{\ep}{\epsilon}
\newcommand{\var}[1]{\delta #1\,}
\newcommand{\dd}[1]{\mathrm{d}#1\,}
\newcommand{\oes}[1]{\textcolor{red}{#1}}
\newcommand{\oesimp}[1]{\textcolor{blue}{#1}}
\newcommand{\vb}{\mathbf}
\newcommand{\ketbra}[2]{\ket{#1}\!\!\bra{#2}}
\newcommand{\Tr}{\operatorname{Tr}}
\newcommand{\up}{\uparrow}
\newcommand{\down}{\downarrow}


\fancyhead[L]{\textbf{Frustrated Magnetism}}


\begin{document}

Lectures by Claudio Castelnovo
\section{Lecture 1}
\subsection{Introduction}
Aims: Brief introduction to fractionalised quasiparticles that can appear in frustrated
magnetism.

Frustrated magnetism: The inability to minimise all terms of Hamiltonian
simultanously, cannot be described by Landau-Ginzburg theory.

Has many interesting properties like
\begin{itemize}
    \item Degeneracy
    \item Emergent Symmetries
    \item Topological Order
    \item Fractional Excitations
\end{itemize}
Lectures will be structured as series of examples.

\subsection{Conventional Magnetism}
We have crystal structure, lattice $\Lambda$ and each site has magnetic moments
$\vb{\mu}_i$. We shall take them to be along 1 axis, and so assume they are of
Ising type with only nearest neigbour interactions.
\[
    H=-J\sum_{\braket{ij}} \sigma_i\sigma_j
\]
where $J>0$ for ferromagnetic interactions, which favour alignment. This is
combatted by entropy, which favours random $\{\sigma_i\}$.
The interactions come about from exchange couplings of neighbouring
electrons\footnote{Limit of Hubbard model? Must investigate}

low sym ~1 High symm phase
--------*-------> T/J

Has $\Z_2$ symmetry in high temperature phase and this symmetry is
\emph{broken}.
\begin{remark}
    Each individual energy term is minimised
\end{remark}

\begin{example}[Spin correlations]
How do we compute correlations? Define partition function
\[ 
    Z=\sum_{\{\sigma_i\}} \e^{-\beta H}
\]
which then allows us to compute correlation

\[
    \braket{\sigma_l\sigma_m}=\frac{1}{Z}\sum_{\{\sigma_i\}}\sigma_l\sigma_k\e^{\beta
    J\sum_{\braket{ij}} \sigma_i\sigma_j}
\]
At high temperature  $\beta J\ll 1$, so that the lowest terms in the taylor
expansion of the probality density dominate, and we obtain 
\[
    \braket{\sigma_l\sigma_k}\simeq\frac{2\beta J_{lm}}{Z}
\]
where $J_{lm}=\begin{cases}J &\text{$l,m$ are n.neighbours}\\
0&\text{otherwise}\end{cases}$ 
These correlations are \emph{trivial} correlation as the spins only reflect the
hamiltonian.
\end{example}

For conventional magnetism, the behaviour is therefore trival everywhere except
near critical point. At low temperature the order is trival; If $\sigma=1$ then
most likely all other $\sigma$ will also be $1$. At high temperature the
correlations are trivial $\braket{\sigma\sigma}\sim\beta H$, where the part of
the hamiltonian refered to is the interaction of the two appropriate spins. The
kind of interesting behaviour seen near critical point
\begin{itemize}
    \item Spin correlations decay with power law
    \item Scale invariance
    \item Critical scaling of physical properties with distance from C.P. in
        parameter space. ($C_V\sim|T-T_C|^\alpha$, $\braket{\sigma}\sim
        B^{1/\delta}$)
    \item Universality
\end{itemize}

\begin{example}[Quantum phase transitions]
    Now assume quantum spins and add a transverse field $\Gamma$ (Quantum phase
    transition arises due to non-commutativity of terms) 
    \[
        H=-J\sum_{\braket{ij}}\sigma_i^Z\sigma_j^Z +\Gamma \sum_i\sigma_i^X.
    \]
   Now 2D parameter space of $T,\Gamma$. phase boundary of ordered and
   disordered phases is now a line, and at $T=0$, we have a Quantum critical
   point. Fluctations near the quantum critical point take much different form
   from usual, as it will be of $d+1$ dimensions in contrast to the classical
   which is at $d$ dimensions. These fluctations spreaed further into the phase
   diagram into a cone in the quantum critical regime.
\end{example}
Conventional magnetism has trivial correlations apart from critical behaviour.
Relies on being minimise all energy terms at the same time.
Instead can have competition between lattice geometry and interaction terms.
\subsection{Frustrated Magnetism}

\[ H=\sum_{ij}H_{ij}\]
such that not all $H_{ij}$ can be minimised at the same time.
This supresses the critcal temperature significantly $T_c/J\ll 1$, where $J$ is
characteristic scale of $H_{ij}$, and in the
region $T_c<T\lesssim J$ non trivial region. One can pseudo quantify the degree
of frustration by the ratio $f=\frac{T_c}{|\Theta_{CW}|}$, of the critical
temperature and the Curie-Weiss temperature. This region is a new phase called a \emph{spin
liquid}.
\begin{remark}[Properties of Spin liquids]
    Spin liquids can both be quantum and classical show 
    \begin{itemize}
        \item No long range order
        \item Strong correlations, as $T<J$
        \item Have large degeneracy/entanglement
        \item Emergent Symmetries
        \item Topological order
        \item Fractionalised Quantum particles.
    \end{itemize}
    and examples of systems exhibiting it include Triangular Ising AFM4, 8-vertex
    model, and the Quantum Toric Code. Note that the frustration does not always
    lead to these properties, but it does allow them. 
    Note that it is only bounded by a critical point on the lower bound in classical
    case, hence it is technically a conventionally defined states. This may not
    be the case for quantum systems, where the system may be
    gapped\footnote{expand on this point?}. Note theoretical models can be
    \emph{fully frustrated} with $f=0$, though these are not physical, as they
    have finite 0 temperature entropy.
\end{remark}
\begin{example}[Triangular Ising AFM]
    A completely classical anti-ferromagnetic model 
    \[ H= H\sum_{\braket{ij}}\sigma_i\sigma_j\]
    with $J>0$. On triangular lattice, cannot minimise all terms. Consider
    a triangle; We can at most have 2 anti-ferromagnetic bonds for every 3, and
    once two spins have been chosen, have degeneracy for the choice of the
    third. These are \emph{2-1 triangles} and can be used to tile the whole
    lattice. In particular, can label the ferromagnetic bond of each triangle by
    a dimer going between the centers of the two triangles which the bond
    touches. This gives a $2:1$ mapping of spins to dimers,called \emph{the
    dimer model}.
    %TODO:Create figures
    \begin{remark}
    For square lattice, can easily minimise all energy terms by making them
    anti-ferromagnetic.
    \end{remark}
\end{example}
\end{document}
