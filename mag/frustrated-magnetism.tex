\documentclass[a4paper]{article}
\author{Oscar Emil Sommer}
\makeatletter
\usepackage{amsfonts}
\usepackage{amsmath}
\usepackage{amssymb}
\usepackage{amsthm}
\usepackage{bm}
\usepackage[english]{babel}
\usepackage{booktabs}
\usepackage[margin = 10pt, font = small, labelfont = bf]{caption}
\usepackage{fancyhdr}
\usepackage{graphicx}
\usepackage{xcolor}
\usepackage{mathtools}
\usepackage{microtype}
\usepackage{multirow}
\usepackage{pdflscape}
\usepackage{pgfplots}
\usepackage{siunitx}
\usepackage{tabularx}
\usepackage{tikz}
\usepackage{sectsty}
\usepackage{derivative}
\usepackage{braket}
\usepackage[colorlinks=true
   ,urlcolor=blue
   ,anchorcolor=blue
   ,citecolor=blue
   ,filecolor=blue
   ,linkcolor=blue
   ,menucolor=blue
   ,linktocpage=true
   ,pdfa=true
]{hyperref}

\pagestyle{fancyplain}
\fancyhead[R]{Oscar Emil Sommer}
\fancyfoot{}
\fancyfoot[C]{\thepage}

\renewcommand{\@dotsep}{10000} %No dots in ToC
\allsectionsfont{\bfseries\sffamily} %Sections are bold sans serif



% Theorem-like environments
\theoremstyle{definition}
\newtheorem*{axiom}{Axiom}
\newtheorem*{aprx}{Approximation}
\newtheorem*{claim}{Claim}
\newtheorem{theorem}{Theorem}[section]
\newtheorem{corollary}[theorem]{Corollary}
\newtheorem{definition}{Definition}[section]
\newtheorem{conjecture}{Conjecture}
\newtheorem*{example}{Example}
\newtheorem*{exercise}{Exercise}
\newtheorem*{fact}{Fact}
\newtheorem*{remark}{Remark}
\newtheorem*{method}{Method}
\newtheorem{lemma}[theorem]{Lemma}
\newtheorem{proposition}[theorem]{Proposition}


\newtheorem*{principle}{Principle}
\setcounter{tocdepth}{2}
\makeatother

%Upright symbols
\renewcommand{\i}{\mathrm{i}}
\newcommand{\e}{\mathrm{e}}
\newcommand{\p}{\partial}
%Vector formatting
\renewcommand{\Vec}[2]{\left(\begin{array}{c} {#1} \\ {#2} \end{array}\right)}


%Common sets
\newcommand{\R}{\mathbb{R}}
\newcommand{\Z}{\mathbb{Z}}
\newcommand{\N}{\mathbb{N}}
\newcommand{\Q}{\mathbb{Q}}
\newcommand{\C}{\mathbb{C}}
\newcommand{\M}{\mathcal{M}}

%Common Groups
\newcommand{\SU}{\mathrm{SU}}
\newcommand{\SL}{\mathrm{SL}}
\newcommand{\GL}{\mathrm{GL}}
\newcommand{\Orth}{\mathrm{O}}
\newcommand{\SO}{\mathrm{SO}}
\newcommand{\U}{\mathrm{U}}
\renewcommand{\Im}{\operatorname{Im}}
\renewcommand{\Re}{\operatorname{Re}}
\renewcommand{\O}{\mathcal{O}}

\newcommand{\su}{\mathfrak{su}}
\newcommand{\gl}{\mathfrak{gl}}
\newcommand{\so}{\mathfrak{so}}
%Common stylistic variable

\newcommand{\ep}{\epsilon}
\newcommand{\var}[1]{\delta #1\,}
\newcommand{\dd}[1]{\mathrm{d}#1\,}
\newcommand{\oes}[1]{\textcolor{red}{#1}}
\newcommand{\oesimp}[1]{\textcolor{blue}{#1}}
\newcommand{\vb}{\mathbf}
\newcommand{\ketbra}[2]{\ket{#1}\!\!\bra{#2}}
\newcommand{\Tr}{\operatorname{Tr}}
\newcommand{\up}{\uparrow}
\newcommand{\down}{\downarrow}


\fancyhead[L]{\textbf{Frustrated Magnetism}}


\begin{document}

Lectures by Claudio Castelnovo
\section{Lecture 1}
\subsection{Introduction}
Aims: Brief introduction to fractionalised quasiparticles that can appear in frustrated
magnetism.

Frustrated magnetism: The inability to minimise all terms of Hamiltonian
simultanously, cannot be described by Landau-Ginzburg theory.

Has many interesting properties like
\begin{itemize}
    \item Degeneracy
    \item Emergent Symmetries
    \item Topological Order
    \item Fractional Excitations
\end{itemize}
Lectures will be structured as series of examples.

\subsection{Conventional Magnetism}
We have crystal structure, lattice $\Lambda$ and each site has magnetic moments
$\vb{\mu}_i$. We shall take them to be along 1 axis, and so assume they are of
Ising type with only nearest neigbour interactions.
\[
    H=-J\sum_{\braket{ij}} \sigma_i\sigma_j
\]
where $J>0$ for ferromagnetic interactions, which favour alignment. This is
combatted by entropy, which favours random $\{\sigma_i\}$.
The interactions come about from exchange couplings of neighbouring
electrons\footnote{Limit of Hubbard model? Must investigate}

low sym ~1 High symm phase
--------*-------> T/J

Has $\Z_2$ symmetry in high temperature phase and this symmetry is
\emph{broken}.
\begin{remark}
    Each individual energy term is minimised
\end{remark}

\begin{example}[Spin correlations]
How do we compute correlations? Define partition function
\[ 
    Z=\sum_{\{\sigma_i\}} \e^{-\beta H}
\]
which then allows us to compute correlation

\[
    \braket{\sigma_l\sigma_m}=\frac{1}{Z}\sum_{\{\sigma_i\}}\sigma_l\sigma_k\e^{\beta
    J\sum_{\braket{ij}} \sigma_i\sigma_j}
\]
At high temperature  $\beta J\ll 1$, so that the lowest terms in the taylor
expansion of the probality density dominate, and we obtain 
\[
    \braket{\sigma_l\sigma_k}\simeq\frac{2\beta J_{lm}}{Z}
\]
where $J_{lm}=\begin{cases}J &\text{$l,m$ are n.neighbours}\\
0&\text{otherwise}\end{cases}$ 
These correlations are \emph{trivial} correlation as the spins only reflect the
hamiltonian.
\end{example}

For conventional magnetism, the behaviour is therefore trival everywhere except
near critical point. At low temperature the order is trival; If $\sigma=1$ then
most likely all other $\sigma$ will also be $1$. At high temperature the
correlations are trivial $\braket{\sigma\sigma}\sim\beta H$, where the part of
the hamiltonian refered to is the interaction of the two appropriate spins. The
kind of interesting behaviour seen near critical point
\begin{itemize}
    \item Spin correlations decay with power law
    \item Scale invariance
    \item Critical scaling of physical properties with distance from C.P. in
        parameter space. ($C_V\sim|T-T_C|^\alpha$, $\braket{\sigma}\sim
        B^{1/\delta}$)
    \item Universality
\end{itemize}

\begin{example}[Quantum phase transitions]
    Now assume quantum spins and add a transverse field $\Gamma$ (Quantum phase
    transition arises due to non-commutativity of terms) 
    \[
        H=-J\sum_{\braket{ij}}\sigma_i^Z\sigma_j^Z +\Gamma \sum_i\sigma_i^X.
    \]
   Now 2D parameter space of $T,\Gamma$. phase boundary of ordered and
   disordered phases is now a line, and at $T=0$, we have a Quantum critical
   point. Fluctations near the quantum critical point take much different form
   from usual, as it will be of $d+1$ dimensions in contrast to the classical
   which is at $d$ dimensions. These fluctations spreaed further into the phase
   diagram into a cone in the quantum critical regime.
\end{example}
Conventional magnetism has trivial correlations apart from critical behaviour.
Relies on being minimise all energy terms at the same time.
Instead can have competition between lattice geometry and interaction terms.
\subsection{Frustrated Magnetism}

\[ H=\sum_{ij}H_{ij}\]
such that not all $H_{ij}$ can be minimised at the same time.
This supresses the critcal temperature significantly $T_c/J\ll 1$, where $J$ is
characteristic scale of $H_{ij}$, and in the
region $T_c<T\lesssim J$ non trivial region. One can pseudo quantify the degree
of frustration by the ratio $f=\frac{T_c}{|\Theta_{CW}|}$, of the critical
temperature and the Curie-Weiss temperature. This region is a new phase called a \emph{spin
liquid}.
\begin{remark}[Properties of Spin liquids]
    Spin liquids can both be quantum and classical show 
    \begin{itemize}
        \item No long range order
        \item Strong correlations, as $T<J$
        \item Have large degeneracy/entanglement
        \item Emergent Symmetries
        \item Topological order
        \item Fractionalised Quantum particles.
    \end{itemize}
    and examples of systems exhibiting it include Triangular Ising AFM4, 8-vertex
    model, and the Quantum Toric Code. Note that the frustration does not always
    lead to these properties, but it does allow them. 
    Note that it is only bounded by a critical point on the lower bound in classical
    case, hence it is technically a conventionally defined states. This may not
    be the case for quantum systems, where the system may be
    gapped\footnote{expand on this point?}. Note theoretical models can be
    \emph{fully frustrated} with $f=0$, though these are not physical, as they
    have finite 0 temperature entropy.
\end{remark}
\subsection{Example: Triangular Ising AFM (Both Lecture 1 and 2)}
A completely classical anti-ferromagnetic model 
\[ H= J\sum_{\braket{ij}}\sigma_i\sigma_j\]
with $J>0$. On triangular lattice, cannot minimise all terms. Consider
a triangle; We can at most have 2 anti-ferromagnetic bonds for every 3, and
once the ferromagnetic bond has been chosen, there is two-fold spin
degeneracy. These are \emph{2-1 triangles} and can be used to tile the whole
lattice. 

\begin{remark}
For square lattice, can easily minimise all energy terms by making them
anti-ferromagnetic.
\end{remark}
In particular, can label the ferromagnetic bond of each triangle by
a \emph{dimer} going between the centers of the two triangles which the bond
touches. This gives a $2:1$ mapping of spins to dimers,called \emph{the
dimer model}. The centers of the triangular form a honeycomb \emph{dual
lattice} which is where the dimers live. Low energy state has one dimer touching
each center. Can also have high energy excitation with 3 ferromagnetic bonds,
and so 3 dimers touching a center.

To minimise energy, we need exactly one dimer to touch each center, which is
possible at least one way. In fact there are many such ways; take any
closed (or ending at the boundary of your lattice) path of alternating
dimers and non-dimers, and reverse whether each connection is a dimer or
not. This generates an equivalent ground state. Given the number of such
possible paths, the total ground state degeneracy is hence exponential in
the lattice size; so the ground state entropy is exponential in the number
of spins. This is a \emph{fullly frustrated system}. Note there is a way to
solve for the degeneracy exactly in the thermodynamic limit.

\begin{definition}[Full frustration]
A fully frustrated system has an exponential (in system size) number of equivalent ground
states.
\end{definition}

Now we can reformulate again in a flux language. Divide honeycomb lattice into
2 sublattices $A$ and $B$. Let all dimers carry 2 flux units; for $A$ they go
into $B$, and for $B$ they emerge from $A$. The remaining lines carry 1 flux
unit so as to make the fluxes divergenceless. We may therefore think of our
system as a discretised divergenceless field, like that in magnetostatics, which
has an \emph{emergent Gauge Symmetry} (Coulomb phases) [Henley, Ann.Rev.Cond.Mat.2010]. We know how to handle
correlators for the magnetostatic case, so if we can back propagate, we may
solve our original problem. From Monte-Carlo simulations, this works at long distances!

Note that we have had an emergent Gauge Symmetry because we have effectively
remove a large portion of the high-excitation states. This reduced statespace
now has a new emergent symmetry. This is completely different from the
mechanism of symmetry breaking.
%TODO:Create figures

Can we explicitly write Hamiltonian in Dimer Language?
\begin{align*}
    H&=+J\sum_{\braket{ij}}\sigma_i\sigma_j\\
     &=H_0
\end{align*}
Is constant in Dimer Language for $2-1$ triangle.  

For any biparte lattice, can place dimers to get an emergent Electromagnetic
gauge symmetry. 

Can we have resonating dimers? Yes Rokhsar-Kivelson model.
\section{Lecture 2}
\subsection{8 Vertex model}
Simple model for discussing excitations. Classical model of ising spins with Hamiltonian
(note on figures, red is up, blue is down). The lattice sites are $S$, the spin
sites will be $i,j$, and center of plaquetes will be $P$. The Hamiltonian is 
\[
    H=-\lambda \sum_{S}\prod_{i\in S}\sigma_i
\]
with $\lambda>0$.
So product of $4$ spins sorrounding each site. This makes the model less
physical, as it is a four body interaction.

The lowest energy state $\prod_{i\in S} \sigma_i=+1$, which there are 8 possible
minimal states per vetex (Hence 8 vertex model). In total there is exponentially
many ways of arranging them on lattice, so ground state entropy is finite. 
\begin{example}
    Can you find exact degeneracy?
\end{example}
Note there is complete $\Z_2$ symmetry.

\begin{example}[Pauling's estimate of degeneracy]
    Number of ground states can be estimated as 
    \[
        N_\mathrm{GS}=\text{Number of states}\times \text{reduction
            factor}^{\text{Number of terms in hamiltonian}}=
        2^N\cdot\left(\frac{8}{16}\right)^{N/2}
    \]
    So entropy is $\sim \frac{N}{2}\ln 2$. This is a usually an informative
    estimate. What does it neglect? Correlations between constraints, means that
    we overcount.
\end{example}
It turns out that this estimate is exact as correlations are zero length in the
ground state.

Can do better! For any plaquette flip all spins sorrounding it; this flips two
spins for all vertices. This generates a new ground state! We can start with the
ground state with all spin up. Now we can flip any of the $N/2$ plaquettes to
generate all possible ground states. In total $2^{N/2}$ possible operations, but
they are not always independent. For example, assuming periodic boundary
conditions, flipping all all but one plaquette is equivalent to flipping that
plaquette, so there is a total of $2^{N/2-1}$. There are two further independent
operations, called \emph{winding loop flips}, which flips horisontal or vertical
lines of spins. These give exactly $2^{N/2+1}$ ground states.

The exactness of our estimate, the is due to a coincidence due to zero range
correlations.

\begin{align*}
    \braket{\sigma_l\sigma_m}_{\mathrm{GS}}\simeq \sum_{\{\sigma_i\}}
    \sigma_l\sigma_m\\
    &=\sum_{\{\sigma_i\}}
    \frac{1}{2}\left[\sigma_l\sigma_m+\tilde{\sigma}_l\tilde{\sigma}_m \right]\\
    &=0
\end{align*}
where the tilde are obtained by filpping the plaquette that includes $l$ and not
$m$.
This is not a totally trivial model though. In particular, there are
\emph{topological correlations} of \emph{Wilson loops}. In particular, let
$\gamma_{h,v}$ be horisontal or vertical loops of spins, passing through
plauqetes. The product of spins
along the lines 
\[ \Gamma_{h,v}=\prod_{i\in\gamma_{h,v}} \sigma_{i}=\pm 1\]
This is invariant under plaquette flips, but only observable if you know all
spins along lines. This is non-local order, which divides  the config space into
4 topological sectors. Note that it does not matter which line we choose; only
the winding number matters.
\subsubsection{Excitations}
If we flip a single spin we create two excited vertices. Now can move these
around independently by the plaquette flip operations. Now the creates a string,
which  costs energy on at the end points. However, we can move the string around
without energy cost, and is not actually detectible. Hence we have
\emph{fractionalised} our spins into two excitations.

\begin{example}[Conventional vs frustrated magnetism]
    Conventional: Excitations are domain walls
    Frustrated are point like QP on random walks, which have pair
    creation/annihalation.
\end{example}
\subsection{The Quantum Toric Code}
Now put spin-$1/2$ on each bond of a 2D square lattice. Let $S$ be all spins
around a vertex and $P$ all spins around a Plaquette. Now define operators
\begin{align*}
    A_S&=\prod_{i\in S} \sigma_{i}^z\\
    B_P&=\prod_{i\in P} \sigma_i^x
\end{align*}
Now $B_P$ effectively implements the plaquette flips from before. Let the
Hamiltonian be
\[
    H=-\Delta_S\sum_S A_S - \Delta_P\sum_P B_P
\]
with both constants $\Delta_{S,P}>0$. All $A,B$ commute, so in fact the ground
state is eigenstate of all $A,B$. Hence the ground state satisfies
\[
    A_S\ket{\phi_0}=B_P\ket{\phi_0}=\ket{\phi_0}
\]
Now the ground state can be found as as a superposition of all ground states of
$A$; and these can be built of $\ket{1\dots1}$ by application of $B_p$. So if we
project out $-1$ eigenvalues of all $B_P$, we obtain the ground state
\[
    \ket{\phi_0} \sim \prod_P (1+B_P)\ket{1\dots1}
\]
Is the ground state of unique? No!
Think about the winding loop flips from the classical case. Let $\gamma_{h,v}$
be line of spins along edge (note different convention to classical case), and
let
\[ \Gamma_{h,v}=\prod_{i\in \gamma_{v,h}}\sigma_i^x\]
These comute with $A,B$ so they are simultanously diagonalisable and have
eigenvalues $\pm 1$. Hence they define $2^2=4$ ground states
$\ket{\phi_0^{(m_h,m_v)}}$. These $m_{h,v}$ are \emph{topological quantum
numbers}! The $\Gamma$ are called \emph{Wilson loop operators}. It is not
possible to locally distinguish between topological sectors.

Now define line through plaquettes and spins $\tilde{\gamma}_{h,v}$, and define 
\[\tilde{\Gamma}_{h,v}=\prod_{i\in \tilde{gamma}_{h,v}}\sigma_i^z\]
These also commute with $A,B,H$ but not with $\Gamma_{h,v}$. Rather they
anti-commute! The information
encoded by
$(\Gamma_h,\tilde{\Gamma}_v)$ and $(\Gamma_{v},\tilde{\Gamma}_h)$ are equivalent
to a spin-$1/2$ algebra; hence the toric code GS encodes 2 toplogical qubits.
\end{document}
